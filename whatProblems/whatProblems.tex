\documentclass{ximera}

\title{What problems are we solving}

\outcome{Understand what problems currently exist.}

\begin{document}
\begin{abstract}
  Ximera seeks to solve several problems encountered in teaching and
  learning.
\end{abstract}
\maketitle

There are several issues that we have identified with the current
textbook and homework system.

\paragraph{Annual crashes and bugs}

The volume of students who take calculus 1 in the fall places enormous
stress on the servers provided by commercial publishers. For the last
several years, the online homework system for calculus has failed
within the first weeks of the course.  Moreover, commercial publishers
are slow to fix bugs/typos. The current solution for dealing with a
bug in the online homework is to ``not assign'' the faulty problem.


\paragraph{Adhering to other's content}

At The Ohio State University, we have a number of faculty who are
content experts. Commercial solutions place the view of others (often
less qualified) ahead of our own experts. 

Instructor and coordinator control over content; ``no one wants to give someone else's speech''

\paragraph{Alignment of content to assessments}

While commercial solutions often advertise their abundance of practice
problems, the quality of such problems is unclear. In particular, the
practice provided by commercial solutions may or may not align with
our goals for the course. This can create a misalignment of commercial
homework problems with the content of our exams.


\paragraph{New additions}

Publishers release new editions at an approximate rate of a new
edition every 4 years. Each addition brings work for the instructors
and coordinators. This work is typically mundane and
superficial. Instructors should be thinking about content, not section
numbers.


\paragraph{Outsourcing assessment}
Outsourcing assessment is outsourcing a core competency of the University


\paragraph{High costs}
While the mathematics department has done an excellent job negotiating
a lower price for textbooks, the cost is still quite high.  The least
expensive option for a students is to ``rent'' an online copy of the
textbook for \$30 per semester. 


\paragraph{Longevity}
Longevity (can students keep the textbook after the course is over)


\paragraph{Opaqueness}
The transparency (or rather opaqueness) of adaptive grading in commercial platforms

\paragraph{Interaction data is unavailable}

With commercial solutions, we as consumers have little data on how our
students actually use the system. Access to raw data, along with the
ability to gather different and new types of data is needed. With
time, this data will allow us to do predictive analytics.

%% Value of student grade data: we can't get all the data from the
%% commercial publishers, but with our own platform, we get the data
%% needed to do predictive analytics

\end{document}
